\documentclass[11pt]{article}
\input{p.tex}

\title{Transfer-PAC Learning}


\begin{document}
\maketitle

\begin{abstract}
% Description of Transfer Learning
Transfer Learning is the problem of learning from instances drawn from a training distribution and evaluating what is learned on separate instances drawn from a different distribution.

% Description of PAC
PAC is a framework for quantifying the learnable.

% Proposal to combine the two
In this paper, we develop a framework that poses the transfer learning problem as an extension of the PAC framework.

% One sentence explaining why we might want this?

% Explanation of theoretical results achieved, what problems are transferrable, etc.

\end{abstract}

% --- Introduction ---
\section{Introduction}

% --- Transfer Learning ---
\section{Transfer Learning}

% --- PAC Learning ---
\section{PAC Learning}

% --- TPAC Learning ---
\section{Transfer-PAC Learning}

% --- Example: TPAC for Images ---
\section{Example: TPAC for Image Classification}

% --- Example: TPAC for RL/Planning
\section{Example: TPAC for RL/Planning}

Problem Distribution is Parameterized OO-MDP.

Two distributions: $D_{train}$ and $D_{test}$. If the problems are completely unrelated, we get nothing.

Some interesting questions:
\begin{enumerate}
\item What if the action sets are the same or similar?
\item What if the object classes are the same or similar? (perhaps different attribute ranges or something)
\item What if the reward function is the same or similar?
\item What if the transition dynamics are the same or similar?
\end{enumerate}


\end{document}